  \documentclass{beamer}
  \usepackage[utf8]{inputenc}
  \usetheme{Warsaw}

  \title{CQRS}
  \author{Francis Laforge}
  \institute{Introduction au pattern CQRS}

  \begin{document}

  \begin{frame}
  \titlepage
  \end{frame}

  \frame{\frametitle{Sommaire}\tableofcontents} 

	\section{Définition}
	\begin{frame}
		\begin{itemize}
	    \item \textsc{CQRS} : \textbf{C}ommand \textbf{Q}uery \textbf{R}esponsibility \textbf{S}egregation
	    \item Séparation de la partie écriture (commands ou encore modifiers, mutators) et lecture (query)
	    \item Command : create, update, delete
	    \item Query: read
		\end{itemize}
	\end{frame}

  \section{Pourquoi}
  \begin{frame}
  		\begin{itemize}
	  		\item Responsabilité
	  		\item Une "command" induit des modifications
	  		\item Les modèles peuvent différer entre "command" et "query"
  		\end{itemize}
	\end{frame}

  \section{Quand}
  \begin{frame}
  	Quand l'utiliser :
	\begin{itemize}
	  	\item 80 / 20
	  	\item Différents data engine
	  	\item Besoin de scale, cache, transaction ...
		\item Système complexe
    \end{itemize}
	Beaucoup de cas où il ne faut PAS l'utiliser (simple CRUD, modèle partagé ...)
  \end{frame}

  \section{Exemple}
  \begin{frame}
	Un exemple d'application \href{https://github.com/FrancisLfg/photographart}{ici} + schéma
  \end{frame}

  \section{Conclusion}
  \begin{frame}
  	CQRS fonctionne bien avec :
	\begin{itemize}
		\item DDD
		\item Command pattern
		\item Event sourcing
	\end{itemize}
	Souvent eux-même confondus avec CQRS...
   \end{frame}
   \begin{frame}
	Amène de la clarté dans certains cas et peut-être utilisé dans un contexte isolé.
	\\ Framework : 
	\begin{itemize}
		\item AxonFramework (Java, 2k stars sur github)
		\item wolkenkit (Typescript, 1k stars sur github)
		\item many others
	\end{itemize}
  \end{frame}
  \end{document}